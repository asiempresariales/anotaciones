Links pruebas locales
********************************************************************************************************************
Swanger

http://localhost:8080/swagger-ui/index.html#

********************************************************************************************************************
Los repositories de spring son interfaces que extienden de JpaRepository
o de CrudRepository<Product, Long>
se les pasa el tipo de entidad y el tipo de la llave primaria
Aunque no se anota la interface con ninguna anotacion, es detectado como un componente de spring
al extender de JpaRepository o CrudRepository

********************************************************************************************************************
Los servicios son interfaces que nos indican que metodos se deben implementar

La implementacion de los servicios son clases que deben ir anotadas con @Service
ademas  son anotados con @Trasactional(readOnly = true) 
********************************************************************************************************************
H2 base de datos 

http://localhost:8080/h2-console

spring.datasource.url=jdbc:h2:mem:customer
spring.datasource.driverClassName=org.h2.Driver
spring.datasource.username=sa
spring.datasource.password=qwerty
spring.jpa.database-platform=org.hibernate.dialect.H2Dialect
spring.h2.console.enabled=true
spring.jpa.show-sql=true



********************************************************************************************************************

import org.springframework.core.env.Environment;

@Autowired
    private Environment env;

    @GetMapping("/check")
    public String check(){

        return "Hola la el valor de l propiedad es"+ env.getProperty("custom.activeprofileName") ;

    }

********************************************************************************************************************
Ejecutar jar con perfil

java -jar  directorio   --spring..active=dev



********************************************************************************************************************


Java runtime vs code 
cmd + shift + p
Java: Configure Java Runtime

********************************************************************************************************************
ENPAQUETAR jar video 218 

mvnm clean package
mvnw package
mvnm clean install es para subir a repositorio


./mvnw clean package



********************************************************************************************************************
CONFIG SERVER - Servidor de configuracion de propiedades se usa para centralizar las propiedades de los microservicios
********************************************************************************************************************

en el archivo principal de la aplicacion habilitar la anotacion @EnableConfigServer

@EnableConfigServer

En todos los microservicios se debe agregar el starter de config server
en todos los microservicios se debe agregar la dependencia de config server

********************************************************************************************************************
pruba cloud - config server - llamando propiedades desde archivo properties externo

http://localhost:8888/config-server-repo/local

host:puerto/nombre de la carpeta del proyecto/perfil 


********************************************************************************************************************
Para iniciar la aplicacion desde un jar con un perfil especifico
application-dev.properties es el nombre del archivo de propiedades
el nombre del archivo es la primer palabra y el perfil es la segunda palabra

java - jar -Dspring.cloud.config.name=application -Dspring.profiles.active=dev spring-boot-config-client-0.0.1-SNAPSHOT.jar



********************************************************************************************************************
.uri con variable de entorno 
spring.cloud.config.uri=${CONFIG_SERVER_URI:http://localhost:8888}

spring.cloud.config.server.git.uri=${CONFIG_SERVER_URI:${user.dir}Documents/udemy/microservicios/caso/config-server-repo}

// Declarar variable de entorno en local 
export REPOSITORY_PATH=/Users/ascencio/Documents/udemy/microservicios/caso/config-server-repo

ver variables de entorno mac: printenv
//para pintar la variable 
echo REPOSITORY_PATH



********************************************************************************************************************
EUREKA SERVER - Servidor de registro y descubrimiento de servicios
********************************************************************************************************************


Eureka server

En archivo principal habilitar la anotacion @EnableEurekaServer

@EnableEurekaServer

En todos los microservicios se debe agregar el starter de eureka server

Para eviatr la configuracion de config server.  en el archivo de propiedades se debe agregar la siguiente linea
spring.cloud.config.enabled=false

********************************************************************************************************************
Actuator - monitoreo de microservicios
Permite monitorear los microservicios en tiempo real y obtener informacion de los mismos

http://192.168.100.74:8082/actuator
http://192.168.100.74:8082/actuator/health
http://192.168.100.74:8082/actuator/metrics


EUREKA REST operation Git con diferentes endpoitns  que se pueden usar para obtener informacion de los microservicios


********************************************************************************************************************
Observabilidad - monitoreo de microservicios
********************************************************************************************************************


Generar microservicio
usar dependencias:

codecentric`s spring boot admin (server)
spring web mvc 

nombre proyecto - Artifacto 
springbootAdmin

clase principal configurar con las siguientes anotaciones 

@Configuration
@EnableAutoConfiguration
@EnableAdminServer
@SpringBootApplication

puort=8062

********************************************************************************************************************


Manejo de excepciones en spring boot

@RestControllerAdvice de aplica a la clase para envolver los metodos de la clase

@ExceptionHandler(Exception.class) se usa para manejar excepciones de cualquier tipo

Por parametros se pasa el tipo de excepcion que se quiere manejar 



********************************************************************************************************************

********************************************************************************************************************
Transacciones
********************************************************************************************************************

Saga Orchestarte Pattern

SagaChoreography pattern



********************************************************************************************************************

********************************************************************************************************************
spring security
********************************************************************************************************************


CSRF - Cross Site Request Forgery
Tecnica en la que un atacante intenta suplantar la identidad de un usuario para realizar acciones no autorizadas
********************************************************************************************************************

CORS - Cross Origin Resource Sharing
origenes cruzados cuando estan diferentes dominios o puertos
********************************************************************************************************************

Configuracion basica restrictiva 

starter security
Tiene configuracion por defecto 

crear package security

crear clase de SecurityConfiguration

import org.springframework.context.annotation.Bean;
import org.springframework.context.annotation.Configuration;
import org.springframework..security.config.Customizer.withDefaults;
import org.springframework.security.config.annotation.web.builders.HttpSecurity;
import org.springframework.security.config.annotation.web.configuration.SecurityFilterChain;
org.springframework.security.web.SecurityFilterChain;

@EnableWebSecurity
@Configuration
public class SecurityConfiguration{

    @Bean
    public SecurityFilterChain filterChain(HttpSecurity http) throws Exception{
        
        http.csrf().disable()   //deshabilitar la proteccion csrf
            .authorizeRequests()    //cualquier peticion http tiene que estar autenticada 
            .anyRequest()           //cualquier tipo de peticion
            .authenticated()        //debe estar autenticada
            .and()                  // y ademas adicional 
            .httpBasic(withDefaults())  // debe realizar una autenticacion basica, le pasamos un parametro withdefauls, (que es?) que vamos a utilizar la configuracion de autenticacion  por defecto que nos ofrece  esta configuracion
            .fomLogin(withDefaults());  // como formulario de login vamos a utilizar la configuracion por defecto que nos ofrece spring security
        
        return http.build();
   
    }




}


********************************************************************************************************************
importar  desde org.springframework.boot.autoconfigure.security.servlet.SecurityAutoConfiguration;

En la clase principal se debe excluir la clase por defecto de seguridad que maneja spring de la siguiente manejar
(exclude = {SecurityAutoConfiguration.class})

********************************************************************************************************************
en properties podemos agregar usuario y contraseña para resplazar los que estan por defecto

spring.security.user.name=admin
spring.security.user.password=qwerty


peticion por terminal 
curl -X 'GET'  -H "Authorization: Basic $(echo -n admin:qwerty | base64)"   'http://localhost:7080/billing'   -H 'accept: */*'
********************************************************************************************************************





